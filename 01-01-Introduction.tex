%-----------------------------------------------------------------------------------------------
\section{Introduction}

    Polytropic        processes---\emph{poly}:        many,        \emph{tropic}:         forms,
    ways~\cite{2020-NaaktgeborenC-engrXiv}---are a versatile class of equilibrium thermodynamics
    processes          with          many          applications          in          engineering
    thermodynamics~\cite{2002-MoranMJ+ShapiroHN-LTC,                 2013-CengelYA+BolesMA-AMGH,
    2015-KroosKA+PotterMC-Cengage,  1986-JonesJB+HawkinsGA-Wiley},  given   by   the   following
    relationship
    %
    \begin{equation}
        Pv^n = \mathsf{c} = \mbox{const.},
        \label{eq:polytropic}
    \end{equation}
    %
    \noindent where $P$ is the system pressure, $v$ the system specific volume, and $n$  is  the
    polytropic exponent.

    Polytropic       processes       are       said       to       find        support        in
    measurements~\cite{2013-CengelYA+BolesMA-AMGH,                   2002-MoranMJ+ShapiroHN-LTC,
    1985-WylenG-Wiley};         however,          theory~\cite{2012-ChristiansJ-IntJMechEngEduc,
    2020-NaaktgeborenC-engrXiv} predicts a fairly restrictive set of conditions---on the process
    boundary and internal conditions and on the underlying substance model---are required for  a
    process to follow Eq.~(\ref{eq:polytropic}) \emph{exactly}---see, for instance, Theorem~3 of
    reference~\cite{2020-NaaktgeborenC-engrXiv}.

    Thus, in non-ideal  settings,  polytropic  processes  either  (i)~hold  approximately  in  a
    \emph{local} neighborhood of  a  given  state,  for  constant-$n$,  or  (ii)~the  polytropic
    exponent must be generalized into a function of a state property, such  as  temperature,  or
    even further generalized.

    Reference~\cite{2020-NaaktgeborenC-engrXiv} brings useful definitions for  discussions  like
    this, such as that of a \emph{logical process},  and  also  those  of  \emph{exact}  and  of
    \emph{local} polytropic processes.

    In \emph{exact polytropic process}, Eq.~(\ref{eq:polytropic}) holds exactly with a constant,
    unique polytropic exponent, for the entire duration of the \emph{logical process}. They  are
    shown to be able to represent any straight line segment process in $\log P  \times  \log  v$
    coordinates~\cite{2020-NaaktgeborenC-engrXiv}.

    Moreover, process whose representations in $\log P \times \log v$  coordinates  are  curved,
    are shown to be able to be  approximated  by  a  finite  number  of  \emph{local  polytropic
    processes} within  finite  error  intervals~\cite{2020-NaaktgeborenC-engrXiv}.  Observations
    like these attest the flexibility of processes based on the  polytropic  relationship,  thus
    justifying the flexibility encoded in their name.

    Despite their enormous flexibility, the defining concept  of  polytropic  processes  can  be
    further generalized as to allow for increased flexibility, so that some  processes  families
    with curved representation in $\log P  \times  \log  v$  coordinates  might  have  an  exact
    representation in such further generalized process relation  concept.  The  introduction  of
    such generalizations can be of theoretical and of applied interest.

    Therefore, this work proposes one such generalization, named \emph{polyekthetic  processes},
    that form a \emph{larger process set} of which  the  polytropic  process  set  is  a  proper
    subset, so that any polytropic process is a particular case of a polyekthetic  process,  but
    not necessarily the opposite.

    The origin  (etymology)  for  the  proposed  `polyekthetic'  term  is  also  given,  and  an
    application yielding exact polyekthetic processes  (but  not  exactly  polytropic  ones)  is
    provided.

%-----------------------------------------------------------------------------------------------

