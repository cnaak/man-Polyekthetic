%-----------------------------------------------------------------------------------------------
\section{Introduction}

    Polytropic        processes---\emph{poly}:        many,        \emph{tropic}:         forms,
    ways~\cite{2020-NaaktgeborenC-engrXiv}---are a versatile class of equilibrium thermodynamics
    processes   with   many   applications   and    experimental    support    in    engineering
    thermodynamics~\cite{2002-MoranMJ+ShapiroHN-LTC,                 2013-CengelYA+BolesMA-AMGH,
    2015-KroosKA+PotterMC-Cengage,  1986-JonesJB+HawkinsGA-Wiley},  given   by   the   following
    relationship
    %
    \begin{equation}
        Pv^n = \mathsf{c} = \mbox{const.},
        \label{eq:polytropic}
    \end{equation}
    %
    \noindent where $P$ is the system pressure, $v$ the system specific volume, and $n$  is  the
    polytropic exponent.

    In   \emph{exact   polytropic   process},   defined    in~\cite{2020-NaaktgeborenC-engrXiv},
    Eq.~(\ref{eq:polytropic}) holds exactly with a constant, unique polytropic exponent, for the
    entire     duration     of      the      \emph{logical      process},      also      defined
    in~\cite{2020-NaaktgeborenC-engrXiv}. They are shown to be able to  represent  any  straight
    line      segment      process       in       $\log       P       \times       \log       v$
    coordinates~\cite{2020-NaaktgeborenC-engrXiv}.

    Moreover, process whose representations in $\log P \times \log v$  coordinates  are  curved,
    are shown to be able to be  approximated  by  a  finite  number  of  \emph{local  polytropic
    processes} withing finite  error  intervals~\cite{2020-NaaktgeborenC-engrXiv}.  Observations
    like these attest the flexibility of process based  on  the  polytropic  relationship,  thus
    justifying its etymology.



%-----------------------------------------------------------------------------------------------

