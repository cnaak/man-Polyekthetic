%-----------------------------------------------------------------------------------------------
\section{General Solution}

    Owing to Definition~\ref{def:polyekthetic} being very broad, the focus  now  shifts  towards
    the more applicable Definition~\ref{def:polyekthetic.exponent}, i.e., in finding expressions
    for constant-property polyekthetic exponents $k_{ij\ell}$.

    %---------------------------------------------------------------------------------------
    \subsection{In Partial Differential Notation}

    Let property $\ell\!:\!\ell(i, j)$; then,  the  total  differential  of  $\ell$  is,  for  a
    const-$\ell$ process:
    %
    \begin{equation}
        d\ell = \parxyz\ell ij\,di + \parxyz\ell ji\,dj = 0
        \label{eq:polyek.dl}
    \end{equation}
    %
    \noindent which must be recast into an ODE  whose  solution,  \emph{at  least  locally},  is
    Eq.~(\ref{eq:polyekthetic}). Isolating $di$:
    %
    \begin{equation}
        di = -\parxyz\ell ji \parxyz i\ell j\,dj,
        \label{eq:polyek.di}
    \end{equation}
    %
    \noindent which can be arranged as:
    %
    \begin{equation}
        \frac{di}{i} = \frac{-j}{i} \parxyz\ell ji \parxyz i\ell j \frac{dj}{j}.
        \label{eq:polyek.ode}
    \end{equation}

    The polyekthetic exponent, $k_{ij\ell}$, is therefore:
    %
    \begin{equation}
        k_{ij\ell} = \frac{j}{i} \parxyz\ell ji \parxyz i\ell j.
        \label{eq:polyek.k.raw}
    \end{equation}

    Using the cyclic relationship:
    %
    \begin{equation}
        -1 = \parxyz ji\ell \parxyz\ell ji \parxyz i\ell j,
        \label{eq:cyclic}
    \end{equation}
    %
    \noindent the polyekthetic exponent can be rewritten as:
    %
    \begin{equation}
        k_{ij\ell} = \frac{-j}{i} \parxyz ij\ell.
        \label{eq:polyek.k}
    \end{equation}

    Eq.~(\ref{eq:polyek.k}) is thus the general solution for $k_{ij\ell}$.

    %---------------------------------------------------------------------------------------
    \subsection{In Terms of Bridgman's Relations}

    Perhaps the easiest way of expressing any polyekthetic  exponent  $k_{ij\ell}$  (for  simple
    properties $\ell$) in terms of quantities measurable in laboratory is  by  rewriting  it  in
    terms of Bridgman's relations~\cite{2006-BejanA-Wiley}, which are expressed in  terms  of  a
    peculiar notation.
    %
    \begin{equation}
        \frac{\bri i\ell}{\bri j\ell} \equiv \parxyz ij\ell.
        \label{eq:Bridgman}
    \end{equation}

    Please note that the $\equiv$ sign on Eq.~(\ref{eq:Bridgman}) indicates  the  definition  of
    the \emph{ratio} between shown Bridgman's primitives $\bri i\ell$ and $\bri j\ell$ in  terms
    of $\inlxyz ij\ell$, rather than the other way around.

    Bridgman's relations are tabulated, and can be found on reference~\cite{2006-BejanA-Wiley}.

    The general expression for $k_{ij\ell}$, using Bridgmann's relations notation, is therefore:
    %
    \begin{equation}
        k_{ij\ell} = \frac{-j}{i}\frac{\bri i\ell}{\bri j\ell}.
        \label{eq:k.bri}
    \end{equation}

    %---------------------------------------------------------------------------------------
    \subsection{Properties of Polyekthetic Exponents}

    \begin{theorem}\label{the:k.recip}
        if  $k_{ij\ell}$  is  a  constant-property  process  polyekthetic  exponent,  then   the
        constant-property process polyekthetic exponent
        %
        \begin{equation}
            k_{ji\ell} = \frac{1}{k_{ij\ell}}.
            \label{eq:k.recipr}
        \end{equation}
    \end{theorem}

    \begin{proof}
        Let  $k_{ij\ell}$  be  a  constant-property   process   polyekthetic   exponent,   then,
        Eq.~(\ref{eq:polyek.k}) holds.

        Applying Eq.~(\ref{eq:polyek.k}) for a $k_{ji\ell}$, i.e.,  with  swapped  $i$  and  $j$
        property indices, gives :
        %
        \begin{align}
            k_{ji\ell} &= \frac{-i}{j} \parxyz ji\ell
                       & \rightharpoondown\\
            k_{ji\ell} &= \left[\frac{-j}{i} \parxyz ij\ell\right]^{-1}
                       & \rightharpoondown\\
            k_{ji\ell} &= \left(k_{ij\ell}\right)^{-1} = \frac{1}{k_{ij\ell}},
        \end{align}
        %
        \noindent thus proving the theorem.
    \end{proof}

    \begin{theorem}\label{the:k.consti}
        All constant-property process polyekthetic exponents $k_{iji} = 0$.
    \end{theorem}

    \begin{proof}
        Let  $k_{ij\ell}$  be  a  constant-property   process   polyekthetic   exponent,   then,
        Eq.~(\ref{eq:polyek.k}) holds.

        Applying  Eq.~(\ref{eq:polyek.k})  for  a  $k_{iji}$,  i.e.,   obtaining   a   const-$i$
        polyekthetic exponent, gives
        %
        \begin{align}
            k_{iji} &= \frac{-j}{i} \parxyz iji
                    & \rightharpoondown\\
            k_{iji} &= 0,
        \end{align}
        %
        \noindent by properties of partial derivatives, thus proving the theorem.
    \end{proof}

    \begin{theorem}\label{the:k.constj}
        All constant-property process polyekthetic exponents $k_{ijj} \to \pm\infty$.
    \end{theorem}

    \begin{proof}
        From Theorem~\ref{the:k.recip}, one has
        %
        \begin{equation}
            k_{ijj} = 1 / k_{jij},
        \end{equation}
        %
        \noindent which, from Theorem~\ref{the:k.consti} which establishes $k_{jij} = 0$,  leads
        to
        %
        \begin{equation}
            k_{ijj} = 1 / 0 \to \pm\infty,
        \end{equation}
        %
        \noindent thus proving the theorem.
    \end{proof}

    Moreover, from Theorems~\ref{the:k.consti} and~\ref{the:k.constj}, whenever property indices
    are repeated, the corresponding polyekthetic exponent $k$ is a trivial one.

    If properties $i$, $j$, and $\ell$, $i \neq j  \neq  \ell  \neq  i$,  are  chosen  from  the
    ``usual'' $\{P, T, v, u, h, s, a, g\}$ engineering thermodynamics  intensive  property  set,
    there will be $8!/5! = 336$ combinations of non-trivial $k_{ij\ell}$. However,  due  to  the
    reciprocity  property  of  Theorem~\ref{the:k.recip},  there  will  be  $168$   non-trivial,
    non-reciprocal $k_{ij\ell}$ required to easily deduce the remaining ones.

%-----------------------------------------------------------------------------------------------

