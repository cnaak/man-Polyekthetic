%-----------------------------------------------------------------------------------------------
\section{General Solution}

    Let $\ell:\ell(i, j)$, the total differential of $\ell$ is, for a const-$\ell$ process:
    %
    \begin{equation}
        d\ell = \parxyz\ell ij\,di + \parxyz\ell ji\,dj = 0
        \label{eq:polyek.dl}
    \end{equation}
    %
    \noindent which must be recast into an ODE whose  solution  is  Eq.~(\ref{eq:polyekthetic}).
    Isolating $di$:
    %
    \begin{equation}
        di = -\parxyz\ell ji \parxyz i\ell j\,dj,
        \label{eq:polyek.di}
    \end{equation}
    %
    \noindent which can be arranged as:
    %
    \begin{equation}
        \frac{di}{i} = \frac{-j}{i} \parxyz\ell ji \parxyz i\ell j \frac{dj}{j}.
        \label{eq:polyek.ode}
    \end{equation}

    The polyekthetic exponent, $k_{ij\ell}$, is therefore:
    %
    \begin{equation}
        k_{ij\ell} = \frac{j}{i} \parxyz\ell ji \parxyz i\ell j.
        \label{eq:polyek.k.raw}
    \end{equation}

    Using the cyclic relationship:
    %
    \begin{equation}
        -1 = \parxyz ji\ell \parxyz\ell ji \parxyz i\ell j,
        \label{eq:cyclic}
    \end{equation}
    %
    \noindent the polyekthetic exponent can be rewritten as:
    %
    \begin{equation}
        k_{ij\ell} = \frac{-j}{i} \parxyz ij\ell.
        \label{eq:polyek.k}
    \end{equation}

    Eq.~(\ref{eq:polyek.k}) is thus the general solution for $k_{ij\ell}$.

    Perhaps the easiest way of expressing any polyekthetic exponent  $k_{ij\ell}$  in  terms  of
    quantities  measurable  in  laboratory  is  by  rewriting  it   in   terms   of   Bridgman's
    relations~\cite{2006-BejanA-Wiley}, which are expressed in terms of a peculiar notation.
    %
    \begin{equation}
        \frac{\bri i\ell}{\bri j\ell} \equiv \parxyz ij\ell.
        \label{eq:Bridgman}
    \end{equation}

    Please note that the $\equiv$ sign on Eq.~(\ref{eq:Bridgman}) indicates  the  definition  of
    the \emph{ratio} between shown Bridgman's primitives $\bri i\ell$ and $\bri j\ell$ in  terms
    of $\inlxyz ij\ell$, rather than the other way around.

    Bridgman's relations are tabulated, and can be found on reference~\cite{2006-BejanA-Wiley}.

    The general expression for $k_{ij\ell}$, using Bridgmann's relations notation, is therefore:
    %
    \begin{equation}
        k_{ij\ell} = \frac{-j}{i}\frac{\bri i\ell}{\bri j\ell}.
        \label{eq:k.bri}
    \end{equation}

    It is straightforward to demonstrate the following properties of $k_{ij\ell}$:
    %
    \begin{align}
        \label{eq:polyek.k.recipr}
        k_{ji\ell}  &= -i\partial j\ell / j\partial i\ell = k_{ij\ell}^{-1},    & \qquad & \mbox{reciprocity,} \\
        \label{eq:polyek.k.consti}
        k_{iji}     &= 0,                                                       & \qquad & \mbox{$i$-constancy, and} \\
        \label{eq:polyek.k.constj}
        k_{ijj}     &= k_{jij}^{-1} = k_{iji}^{-1} = \pm\infty,                 & \qquad & \mbox{$j$-constancy},
    \end{align}
    %
    \noindent in which the last equality comes from the  immateriality  of  the  $i$,  $j$,  and
    $\ell$  property  indices.  Therefore,  whenever  property   indices   are   repeated,   the
    corresponding exponent $k$ is a trivial one.

    If properties $i$, $j$, and $\ell$, $i \neq j \neq \ell$,  are  chosen  from  the  ``usual''
    $\{P, T, v, u, h, s, a, g\}$ engineering thermodynamics intensive property set,  there  will
    be $8!/5! = 336$ combinations of non-trivial $k_{ij\ell}$. However, due to  the  reciprocity
    property  of  Eq.~(\ref{eq:polyek.k.recipr}),   only   $168$   non-trivial,   non-reciprocal
    $k_{ij\ell}$ are required to easily deduce the remaining ones.

%-----------------------------------------------------------------------------------------------

