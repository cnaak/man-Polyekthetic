%-----------------------------------------------------------------------------------------------
\section{General Solution}

    Let $\ell:\ell(i, j)$, the total differential of $\ell$ is, for a const-$\ell$ process:
    %
    \begin{equation}
        d\ell = \parxyz\ell ij\,di + \parxyz\ell ji\,dj = 0
        \label{eq:polyek.dl}
    \end{equation}
    %
    \noindent which must be recast into an ODE whose  solution  is  Eq.~(\ref{eq:polyekthetic}).
    Isolating $di$:
    %
    \begin{equation}
        di = -\parxyz\ell ji \parxyz i\ell j\,dj,
        \label{eq:polyek.di}
    \end{equation}
    %
    \noindent which can be arranged as:
    %
    \begin{equation}
        \frac{di}{i} = \frac{-j}{i} \parxyz\ell ji \parxyz i\ell j \frac{dj}{j}.
        \label{eq:polyek.ode}
    \end{equation}

    The polyekthetic exponent, $k_{ij\ell}$, is therefore:
    %
    \begin{equation}
        k_{ij\ell} = \frac{j}{i} \parxyz\ell ji \parxyz i\ell j.
        \label{eq:polyek.k.raw}
    \end{equation}

    Using the cyclic relationship:
    %
    \begin{equation}
        -1 = \parxyz ji\ell \parxyz\ell ji \parxyz i\ell j,
        \label{eq:cyclic}
    \end{equation}
    %
    \noindent the polyekthetic exponent can be rewritten as:
    %
    \begin{equation}
        k_{ij\ell} = \frac{-j}{i} \parxyz ij\ell.
        \label{eq:polyek.k}
    \end{equation}

    Eq.~(\ref{eq:polyek.k}) is thus the general solution for $k_{ij\ell}$.

    Perhaps the easiest way of expressing any polyekthetic exponent  $k_{ij\ell}$  in  terms  of
    quantities  measurable  in  laboratory  is  by  rewriting  it   in   terms   of   Bridgman's
    relations~\cite{2006-BejanA-Wiley}, which are expressed in terms of a peculiar notation.
    %
    \begin{equation}
        \frac{\bri i\ell}{\bri j\ell} \equiv \parxyz ij\ell.
        \label{eq:Bridgman}
    \end{equation}

    Please note that the $\equiv$ sign on Eq.~(\ref{eq:Bridgman}) indicates  the  definition  of
    the \emph{ratio} between shown Bridgman's primitives $\bri i\ell$ and $\bri j\ell$ in  terms
    of $\inlxyz ij\ell$, rather than the other way around.

    Bridgman's relations are tabulated, and can be found on reference~\cite{2006-BejanA-Wiley}.

%-----------------------------------------------------------------------------------------------

