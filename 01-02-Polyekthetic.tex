%-----------------------------------------------------------------------------------------------
\section{Polyekthetic Processes}

    %---------------------------------------------------------------------------------------
    \subsection{Literature Generalizations of Polytropic Processes}

    Concerning  isentropic  processes  written  in  the   form   of   Eq.~(\ref{eq:polytropic}),
    Bejan~\cite{2006-BejanA-Wiley} derives  a  definition  for  the  \emph{isentropic  expansion
    exponent}, $k$, by differentiating an $s\!:\!s(P, v)$ function and describing the isentropic
    process as:
    %
    \begin{equation}
        ds = \parxyz sPv\,dP + \parxyz svP\,dv = 0.
        \label{eq:iso-sPv}
    \end{equation}

    It is worth noting that Eq.~(\ref{eq:iso-sPv}) is valid  in  general,  and  not  only  to  a
    particular substance equation of state. By a suitable manipulation of the $\inlxyz sPv$  and
    $\inlxyz svP$ coefficients, Eq.~(\ref{eq:iso-sPv}) can be recast into:
    %
    \begin{equation}
        \frac{dP}{P} = -k\frac{dv}{v},
        \label{eq:iso-sPv.ODE}
    \end{equation}
    %
    \noindent with
    %
    \begin{equation}
        k = \frac{-v}{P}\parxyz Pvs.
        \label{eq:iso-sPv.k}
    \end{equation}

    As $k\!:\!k(P, v)$ in general, let $k$ be taken as a  \emph{constant}  in  the  \emph{local}
    vicinity   of   a   state   of   interest;   therefore,   the   \emph{local}   solution   of
    Eq.~(\ref{eq:iso-sPv.ODE}) is the local polytropic process $Pv^k = \mbox{const}$.

    Bejan~\cite{2006-BejanA-Wiley} also derives an \emph{isothermal-expansion exponent},  $k_T$,
    so that $Pv^{k_T}  =  \mbox{const}$  for  isothermal  processes  in  a  similar  fashion  as
    Eqs.~(\ref{eq:iso-sPv})--(\ref{eq:iso-sPv.k}).

    In this type of generalization, one \emph{defines new polytropic  exponents},  so  that  the
    resulting  process,  \emph{written  as  a  polytropic  relationship},  has  a  predetermined
    \emph{behavior} in terms of a non-$Pv$ thermodynamic property. In the given  examples,  this
    last property was either the (i)~entropy, or the (ii)~temperature.

    Moreover,  Nederstigt~\cite{2017-NederstigtP-TUDelft}   generalizes   isentropic   exponents
    $\gamma_{Pv}$, $\gamma_{Tv}$,  and  $\gamma_{PT}$,  so  that  isentropic  processes  can  be
    described as either $Pv^{\gamma_{Pv}} = \mbox{const}$, or $Tv^{\gamma_{Tv}} = \mbox{const}$,
    or $PT^{\gamma_{PT}} = \mbox{const}$.

%-----------------------------------------------------------------------------------------------

