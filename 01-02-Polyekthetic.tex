%-----------------------------------------------------------------------------------------------
\section{Polyekthetic Processes}

    %---------------------------------------------------------------------------------------
    \subsection{Literature Generalizations of Polytropic Processes}

    Concerning  isentropic  processes  written  in  the   form   of   Eq.~(\ref{eq:polytropic}),
    Bejan~\cite{2006-BejanA-Wiley} derives  a  definition  for  the  \emph{isentropic  expansion
    exponent}, $k$, by differentiating an $s\!:\!s(P, v)$ function and describing the isentropic
    process as:
    %
    \begin{equation}
        ds = \parxyz sPv\,dP + \parxyz svP\,dv = 0.
        \label{eq:iso-sPv}
    \end{equation}

    It is worth noting that Eq.~(\ref{eq:iso-sPv}) is valid  in  general,  and  not  only  to  a
    particular substance equation of state. By a suitable manipulation of the $\inlxyz sPv$  and
    $\inlxyz svP$ coefficients, Eq.~(\ref{eq:iso-sPv}) can be recast into:
    %
    \begin{equation}
        \frac{dP}{P} = -k\frac{dv}{v},
        \label{eq:iso-sPv.ODE}
    \end{equation}
    %
    \noindent with
    %
    \begin{equation}
        k \equiv \frac{-v}{P}\parxyz Pvs.
        \label{eq:iso-sPv.k}
    \end{equation}

    As $k\!:\!k(P, v)$ in general, let $k$ be taken as a  \emph{constant}  in  the  \emph{local}
    vicinity   of   a   state   of   interest;   therefore,   the   \emph{local}   solution   of
    Eq.~(\ref{eq:iso-sPv.ODE}) is the local polytropic process $Pv^k = \mbox{const.}$

    Bejan~\cite{2006-BejanA-Wiley} also derives an \emph{isothermal-expansion exponent},  $k_T$,
    so that $Pv^{k_T} = \mbox{const.}$,  for  isothermal  processes  in  a  similar  fashion  as
    Eqs.~(\ref{eq:iso-sPv})--(\ref{eq:iso-sPv.k}).

    In this type of generalization, one \emph{defines new polytropic  exponents},  so  that  the
    resulting  process,  \emph{written  as  a  polytropic  relationship},  has  a  predetermined
    \emph{behavior} in terms of a third thermodynamic property.

    In this ``Bejan'' generalization of polytropic process, one has
    %
    \begin{equation}
        Pv^{k_{\alpha}} = \mathsf{c} = \mbox{const.},
        \label{eq:gen.Bejan}
    \end{equation}
    %
    \noindent for const-$\alpha$ processes.

    Moreover,  Nederstigt~\cite{2017-NederstigtP-TUDelft}   generalizes   isentropic   exponents
    $\gamma_{Pv}$, $\gamma_{Tv}$,  and  $\gamma_{PT}$,  so  that  isentropic  processes  can  be
    described as either $Pv^{\gamma_{Pv}} = \mbox{const}$, or $Tv^{\gamma_{Tv}} = \mbox{const}$,
    or $PT^{\gamma_{PT}} = \mbox{const}$.

    In this ``Nederstigt'' generalization of isentropic processes,  exponents  $k_{\alpha\beta}$
    are such that
    %
    \begin{equation}
        \alpha\beta^{k_{\alpha\beta}} = \mathsf{c} = \mbox{const.},
        \label{eq:gen.Nederstigt}
    \end{equation}
    %
    \noindent for const-$s$ processes.

    %---------------------------------------------------------------------------------------
    \subsection{Polyekthetic Processes Definition}

    The generalization of these ideas follows: let the property indices $i$, $j$, and $\ell$  be
    any one of the usual intensive system properties, then the \emph{exponent}  $k_{ij\ell}$  is
    such that:
    %
    \begin{equation}
        ij^{k_{ij\ell}} = \mathsf{c} = \mbox{const.},
        \label{eq:polyekthetic}
    \end{equation}
    %
    \noindent at least \emph{locally}, for an iso-$\ell$ process.

    Thus, the following are all examples of polyekthetic process descriptions:
    %
    \begin{align}
        Tv^{k_{Tvs}} = \mathsf{c_1} = \mbox{const.}, \label{eq:ex.Tvs} \\
        sP^{k_{sPT}} = \mathsf{c_2} = \mbox{const.}, \label{eq:ex.sPT} \\
        sv^0 = \mathsf{c_3} = \mbox{const.}, \label{eq:ex.sv0} \\
        hv^{\infty} = \mathsf{c_4} = \mbox{const.}, \label{eq:ex.hvh}
    \end{align}
    %
    \noindent in which Eq.~(\ref{eq:ex.Tvs}) describes an isentropic process  in  a  $Tv$  base;
    Eq.~(\ref{eq:ex.sPT}) describes an isothermal process in an $sP$ base; Eq.~(\ref{eq:ex.sv0})
    describes an isentropic process in an $sv$ base, and  Eq.~(\ref{eq:ex.hvh})  an  isenthalpic
    process in an $hv$ base.

    \begin{theorem}\label{the:const-prop}
        Any constant-property process is a polyekthetic process.
    \end{theorem}

    \begin{proof}
        Let  $\ell$  be  the  property  held  constant  for  a  constant-property   process   of
        Theorem~\ref{the:const-prop}.  Then,  by  definition,  the   polyekthetic   process   of
        Eq.~(\ref{eq:polyekthetic}) is a const-$\ell$ process.
    \end{proof}

    Theorem~\ref{the:const-prop} already tokens  polyekthetic  process  are  more  general  than
    polytropic processes, since thet latter cannot specify \emph{any} constant-property process.

    Since polytropic processes are recovered by setting $i \equiv P$, $j \equiv v$, and  letting
    the exponent $k_{Pv\ell} = n$ have any real value, clearly the polytropic process set  is  a
    proper subset of the polyekthetic process set.

    %---------------------------------------------------------------------------------------
    \subsection{Polyekthetic Processes Etymology}

    Following the  ethymology  of  polytropic  processes~\cite{2020-NaaktgeborenC-engrXiv},  and
    given    that    this    generalization    engenders    many    (\GRtxt{pol'u})    exponents
    (\GRtxt{ekj'ethc})~\cite{1997-ManiatoglouMPF-Porto}, the set of processes that  display,  at
    least  locally,  the  relationship  of  the   Eq.~(\ref{eq:polyekthetic})   may   be   named
    ``polyekthetic'', i.e., a \emph{multiple exponent} process set.

%-----------------------------------------------------------------------------------------------

