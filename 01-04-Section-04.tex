%-----------------------------------------------------------------------------------------------
\section{Example Applications}

    %---------------------------------------------------------------------------------------
    \subsection{Iso-$s$ Process in const-$c_v$, van-der-Waals Substance}

    Consider         a         substance         whose         Helmholtz          characteristic
    function~\cite{1986-JonesJB+HawkinsGA-Wiley} in reduced coordinates is given by:
    %
    \begin{equation}
        a_r = \frac{8T_r}{3}\left[
                  \phi\left(1 - \log T_r\right) - \log(3v_r - 1)
              \right] - \frac{3}{v_r},
        \label{eq:ex.aTv}
    \end{equation}
    %
    \noindent where $a_r \equiv a / (P_{cr}v_{cr})$ is the reduced  specific  Helmholtz  energy,
    $T_r \equiv T / T_{cr}$ is the reduced temperature, $\phi \equiv \bar{c}_v / \bar{R}$  is  a
    constant dimensionless isochoric specific heat, $v_r \equiv  v  /  v_{cr}$  is  the  reduced
    specific volume, $P_{cr}$ is the substance critical  pressure,  $T_{cr}$  is  the  substance
    critical temperature, and $v_{cr} = Z_{cr}(RT_{cr}/P_{cr})$ is the substance model  critical
    specific volume, with $Z_{cr} = 3/8$.

    The  Equation  of  State,  EoS,  of  this   substance   is   obtained   by   differentiating
    Eq.~(\ref{eq:ex.aTv}) partially with respect to $v_r$~\cite{1986-JonesJB+HawkinsGA-Wiley}:
    %
    \begin{align}
        P_r & = \parxyz{a_r}{v_r}{T_r} & \rightharpoondown\\
        P_r & = \frac{8T_r}{3v_r - 1} - \frac{3}{v_r^2}, \label{eq:ex.EoS}
    \end{align}
    %
    \noindent where $P_r \equiv P / P_{cr}$ is the reduced pressure, and $P$  the  (dimensional)
    pressure. Eq.~(\ref{eq:ex.EoS}) is  known  as  the  universal~\cite{1899-MaxwellJC-Longmans}
    van-der-Waals~\cite{2006-BejanA-Wiley,                           2013-CengelYA+BolesMA-AMGH,
    1986-JonesJB+HawkinsGA-Wiley} equation of state in reduced coordinates.

    The entropy of this substance is obtained by differentiating Eq.~(\ref{eq:ex.aTv}) partially
    with respect to $T_r$~\cite{1986-JonesJB+HawkinsGA-Wiley}:
    %
    \begin{align}
        s_r & = -\parxyz{a_r}{T_r}{v_r} & \rightharpoondown\\
        s_r & = \frac{8}{3}\left[\phi\log T_r + \log(3v_r - 1)\right], \label{eq:ex.sTv}
    \end{align}
    %
    \noindent where $s_r \equiv s(T_{cr} / P_{cr}v_{cr})$, is the reduced  specific  entropy  of
    the substance and $s$ is the (dimensional) specific entropy.

    Let
    %
    \begin{equation}
        \omega_r \equiv v_r - \frac{1}{3},
        \label{eq:def.omega}
    \end{equation}
    %
    \noindent be the reduced specific voids volume---since the minimum  van-der-Waals  substance
    specific reduced volume, $v_{r,min} = 1/3$, for which $\omega_{r,min} = 0$.

    Therefore, \emph{isentropic processes} in this const-$c_v$, van-der-Waals substance, between
    states $(T_{r1}, \omega_{r1})$ and $(T_{r2}, \omega_{r2})$ are given by:
    %
    \begin{align}
        s_{r2} - s_{r1} & = 0 \qquad\rightharpoondown\\
        \phi\log T_{r2} + \log\omega_{r2} & = \phi\log T_{r1} + \log\omega_{r1},
    \end{align}
    %
    \noindent which, exponentiated and rearranged, leads to
    %
    \begin{equation}
        \omega_{r2}T_{r2}^{\phi} = \omega_{r1}T_{r1}^{\phi},
        \label{eq:vdW.polyek}
    \end{equation}
    %
    \noindent which is a polyekthetic process of constant $ij^n$ with $i \equiv \omega_r$ and $j
    \equiv T_r$ and $n \equiv \phi$.

    Since the process is also  isentropic,  one  has  the  following  constant-property  process
    polyekthetic exponent
    %
    \begin{equation}
        k_{\omega_rT_rs} = \phi \equiv \frac{\bar{c}_v}{\bar{R}},
        \label{eq:vdW.kwTs}
    \end{equation}
    %
    \noindent for this constant-$c_v$, van-der-Waals substance.

    The `K-polytropic' theorem of reference~\cite[p.~8]{2020-NaaktgeborenC-engrXiv} states

    \begin{quote}
        ``Internally reversible processes in constant-specific-heat  unreactive  closed  systems
        with negligible kinetic and potential energy changes  and  constant  heat-to-work  ratio
        interactions, $K$, are exact polytropic processes only if  the  substance  is  an  ideal
        gas.''
    \end{quote}

    Isentropic compressions and expansions have  constant  heat-to-work  ratio  interactions  of
    $K=0$, since isentropic processes are also adiabatic. Therefore, the `K-polytropic'  theorem
    forbids non-ideal substances (such  as  van-der-Waals  ones)  to  display  exact  polytropic
    processes under stated conditions.

    Nonetheless,  Eqs.~(\ref{eq:vdW.polyek})  and~(\ref{eq:vdW.kwTs})   show   that   isentropic
    processes in constant-$c_v$, van-der-Waals  substances  \emph{can}  be  written  as  exactly
    polyekthetic processes with constant polyekthetic exponents.

    This one example illustrates how the polytropic process generalization proposed in this work
    achieved the desired outcome of exactly representing a process that could not be  classified
    as an exact polytropic process~\cite{2020-NaaktgeborenC-engrXiv}.

    %---------------------------------------------------------------------------------------
    \subsection{General Const-Property Polytropic Processes}

    Since  polytropic  processes  are  also  polyekthetic  processes---by  fixing  the  base  as
    $Pv$---one  can  apply  either  Eq.~(\ref{eq:polyek.k})  or~(\ref{eq:k.bri})   in   deriving
    polytropic process exponents  $k_{Pv\ell}$  for  $\ell  \in  \{T,  u,  h,  s,  a,  g\}$  for
    (i)~generic  substance  models,  and  for  (ii)~particular  cases,  such  as  ideal   gases.
    Table~\ref{tab:kPv} brings such values \emph{mostly} in terms of properties that are  easily
    measurable in the laboratory, i.e., in terms of $P$, $T$, $v$, $c_p$, $\beta$, $\kappa$, and
    $\gamma \equiv c_p / c_v$, of which some definitions are given on  Eq.~(\ref{eq:auxs}),  but
    also in terms of the specific entropy $s$:

    \begin{table}[ht]
        \centering
        \caption{General values for $k_{Pv\ell}$ polyekthetic exponents, and for the  ideal  gas
            limit}
        \vspace{4pt}
        \begin{tabular}{ccc}
            \toprule
            Exponent    & General Value &
            Ideal gas limit \\
            \midrule
            $k_{PvP}$   &
            $0$ &
            $0$ \\[\bigskipamount]
            $k_{Pvv}$   &
            $\pm\infty$ &
            $\pm\infty$ \\[\bigskipamount]
            $k_{PvT}$   &
            \(\displaystyle\frac{1}{P\kappa}\) &
            $1$ \\[\bigskipamount]
            $k_{Pvu}$   &
            \(\displaystyle\frac{c_p - \beta Pv}{P(\kappa c_p - T\beta^2v)}\) &
            $1$ \\[\bigskipamount]
            $k_{Pvh}$   &
            \(\displaystyle\frac{c_p}{P[\kappa c_p + \beta v(1 - \beta T)]}\) &
            $1$ \\[\bigskipamount]
            $k_{Pvs}$   &
            \(\displaystyle\frac{c_p}{P(\kappa c_p - T\beta^2v)}\) &
            $\gamma$ \\[\bigskipamount]
            $k_{Pva}$   &
            \(\displaystyle\frac{s + \beta Pv}{P\kappa s}\) &
            \(\displaystyle 1 + \frac{R}{s}\) \\[\bigskipamount]
            $k_{Pvg}$   &
            \(\displaystyle\frac{s}{P(\kappa s - \beta v)}\) &
            \(\displaystyle\frac{s}{s - R}\) \\[\bigskipamount]
            \bottomrule
        \end{tabular}
        \label{tab:kPv}
    \end{table}

    \begin{equation}
        \beta \equiv \frac{1}{v}\parxyz vTP, \qquad
        \kappa \equiv \frac{-1}{v} \parxyz vPT.
        \label{eq:auxs}
    \end{equation}

    Results  show  an  agreement  between  the  polytropic  and  the  corresponding   $Pv$-based
    polyekthetic exponents. The polyekthetic exponent results also show  that  constant-$a$  and
    constant-$g$ process aren't exactly polytropic, even for constant-specific-heat ideal gases,
    given that such exponents are a function of the entropy.

%-----------------------------------------------------------------------------------------------
\section{Conclusions}

    In this work, a generalization of polytropic processes, inspired by their etymology (of many
    ways,  many  forms)  and   by   the   works   of   Bejan~\cite{2006-BejanA-Wiley}   and   of
    Nederstigt~\cite{2017-NederstigtP-TUDelft}, named `polyekthetic' (multi-exponent) processes,
    is proposed.

    It is shown that the  polytropic  process  set  is  a  proper  subset  of  the  polyekthetic
    one---meaning that every polytropic process is also a polyekthetic one but  not  necessarily
    the reverse.

    Moreover, a useful kind  (or  subset)  of  polyekthetic  processes  consisting  of  constant
    arbitrary  property,  and  their  respective  special-notation   $k_{ij\ell}$   polyekthetic
    exponents were defined. Their basic properties were determined, and a general  solution  was
    found   for   them,   stated   either   in   terms   of   general    partial    defivatives,
    Eq.~(\ref{eq:polyek.k}), or of Bridgmann's primitives, Eq.~(\ref{eq:k.bri}).

    An example of exact polyekthetic process that is not an exact polytropic process  was  given
    for a van-der-Waals substance. Moreover, generalized polytropic relationships for \emph{any}
    substance  model,  and  their  corresponding  ideal  gas  limit  values  were  derived  from
    Bridgmann's relations and listed on Table~\ref{tab:kPv}. Results  showed  that  constant-$a$
    and constant-$g$ process aren't exactly polytropic, even  for  constant-specific-heat  ideal
    gases, since the entropy explicitly appears in the polytropic exponent of such processes.

%-----------------------------------------------------------------------------------------------

