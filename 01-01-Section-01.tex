%-----------------------------------------------------------------------------------------------
\section{Introduction}

    Polytropic        processes---\emph{poly}:        many,        \emph{tropic}:         forms,
    ways~\cite{2020-NaaktgeborenC-engrXiv}---are a versatile class of equilibrium thermodynamics
    processes          with          many          applications          in          engineering
    thermodynamics~\cite{2002-MoranMJ+ShapiroHN-LTC,                 2013-CengelYA+BolesMA-AMGH,
    2015-KroosKA+PotterMC-Cengage,  1986-JonesJB+HawkinsGA-Wiley},  given   by   the   following
    relationship
    %
    \begin{equation}
        Pv^n = \mathsf{c} = \mbox{const.},
        \label{eq:polytropic}
    \end{equation}
    %
    \noindent where $P$ is the system pressure, $v$ the system specific volume, and $n$  is  the
    polytropic exponent.

    In   \emph{exact   polytropic   process},   defined    in~\cite{2020-NaaktgeborenC-engrXiv},
    Eq.~(\ref{eq:polytropic}) holds exactly with a constant, unique polytropic exponent, for the
    entire     duration     of      the      \emph{logical      process},      also      defined
    in~\cite{2020-NaaktgeborenC-engrXiv}. They are shown to be able to  represent  any  straight
    line      segment      process       in       $\log       P       \times       \log       v$
    coordinates~\cite{2020-NaaktgeborenC-engrXiv}.

    Moreover, process whose representations in $\log P \times \log v$  coordinates  are  curved,
    are shown to be able to be  approximated  by  a  finite  number  of  \emph{local  polytropic
    processes} within  finite  error  intervals~\cite{2020-NaaktgeborenC-engrXiv}.  Observations
    like these attest the flexibility of processes based on the  polytropic  relationship,  thus
    justifying the flexibility encoded in their name.

    Polytropic process not only find support  in  measurements~\cite{2013-CengelYA+BolesMA-AMGH,
    2002-MoranMJ+ShapiroHN-LTC,  1985-WylenG-Wiley},  but  can  also  be  exactly  derived  from
    theory~\cite{2012-ChristiansJ-IntJMechEngEduc, 2020-NaaktgeborenC-engrXiv}.  In  the  latter
    case, however, the set of conditions is somewhat restrictive---see, for instance,  Theorem~3
    of reference~\cite{2020-NaaktgeborenC-engrXiv}, so that real process with  non-ideal  (real)
    substances are far likely to require \emph{approximation} and \emph{locality} if they are to
    be described by polytropic relationships, even if their  physical  defining  conditions  are
    simple  to  describe---see,  for  instance  the   definition   of   \emph{logical   process}
    in~\cite{2020-NaaktgeborenC-engrXiv}.

    Despite their enormous flexibility, the defining concept  of  polytropic  processes  can  be
    further generalized. This work proposes one such  generalization,  named  \emph{polyekthetic
    processes}, that form a larger process set of which the polytropic process set is  a  proper
    subset, so that any polytropic process is a particular case of a polyekthetic  process,  but
    not the opposite.

    The etymology for the `polyekthetic' term is also given, and an application  yielding  exact
    polyekthetic processes is given.

%-----------------------------------------------------------------------------------------------

